
\subsection{Regression example}

A survey was conducted in 9 areas of the USA to investigate the relationship between
divorce rate (y) and residential mobility (x). Divorce rates in the annual number per 1000 in the population
and the residential mobility is measured by the percentage of the population that moved house in the last
five years.



\begin{tabular}{|c|c|c|c|c|c|c|c|c|c|}
  \hline
Area & 1 & 2 & 3 & 4 & 5 & 6 & 7 & 8 & 9  \\
x & 40 & 38 & 46 & 49 & 47 & 43 & 51 & 57 & 55\\
y & 3.9 & 3.4 & 5.2 & 4.8 & 5.6 & 5.8 & 6.6 & 7.6 & 5.8\\
  \hline
\end{tabular}

\begin{itemize}
\item Check that the following statements are correct.

\begin{itemize}
\item sum of x data = 426
\item sum of squares of x data = 20494
\item sum of y data = 48.7
\item sum of squares of y data = 276.81
\item sum of products of x and y data = 2361
\end{itemize}

\item Derive the estimates for the slope and intercept of the regression line.
\item Estimate the divorce rate for areas that has a residential mobility of 39 and 60 respectively.
\item Which of these estimates is likely to be more accurate? Why?

\end{itemize}
\newpage

\subsection{Regression example}

In a medical experiment concerning 12 patients with a certain type of ear condition,
the following measurements were made for blood flow (y) and auricular pressure (x):

\begin{verbatim}
x<-c(8.5, 9.8, 10.8, 11.5, 11.2, 9.6, 10.1, 13.5, 14.2, 11.8, 8.7, 6.8)
y<-c(3 ,12, 10, 14, 8 ,7 ,9 ,13, 17, 10, 5 ,5)
\end{verbatim}


(Sx =126.5 Sxx =1,381.85 Sy =113 Syy =1251 Sxy =1272.2)


\begin{itemize}
\item Calculate the equation of the least-squares fitted regression line of blood flow
on auricular pressure.
\item Confirm the following values: Sx =126.5, Sxx =1381.85, Sy =113, Syy =1251, Sxy =1272.2.
\item Calculate the correlation coefficient.

\begin{verbatim}
> cor(x,y)
[1] 0.8521414
\end{verbatim}
\end{itemize}


\end{document}