%- http://idatasciencer.com/2015/04/regression-exercise-for-learners/

LOAD IN THE EVALS DATA SET WITH THE HELP OF THE PROVIDED FUNCTION.
load(url("http://bit.ly/1D3dd1n"))
summary(evals)
##      score                 rank            ethnicity      gender   
##  Min.   :2.300   teaching    :102   minority    : 64   female:195  
##  1st Qu.:3.800   tenure track:108   not minority:399   male  :268  
##  Median :4.300   tenured     :253                                  
##  Mean   :4.175                                                     
##  3rd Qu.:4.600                                                     
##  Max.   :5.000                                                     
##         language        age        cls_perc_eval     cls_did_eval   
##  english    :435   Min.   :29.00   Min.   : 10.42   Min.   :  5.00  
##  non-english: 28   1st Qu.:42.00   1st Qu.: 62.70   1st Qu.: 15.00  
##                    Median :48.00   Median : 76.92   Median : 23.00  
##                    Mean   :48.37   Mean   : 74.43   Mean   : 36.62  
##                    3rd Qu.:57.00   3rd Qu.: 87.25   3rd Qu.: 40.00  
##                    Max.   :73.00   Max.   :100.00   Max.   :380.00  
##   cls_students    cls_level      cls_profs         cls_credits 
##  Min.   :  8.00   lower:157   multiple:306   multi credit:436  
##  1st Qu.: 19.00   upper:306   single  :157   one credit  : 27  
##  Median : 29.00                                                
##  Mean   : 55.18                                                
##  3rd Qu.: 60.00                                                
##  Max.   :581.00                                                
##   bty_f1lower     bty_f1upper     bty_f2upper      bty_m1lower   
##  Min.   :1.000   Min.   :1.000   Min.   : 1.000   Min.   :1.000  
##  1st Qu.:2.000   1st Qu.:4.000   1st Qu.: 4.000   1st Qu.:2.000  
##  Median :4.000   Median :5.000   Median : 5.000   Median :3.000  
##  Mean   :3.963   Mean   :5.019   Mean   : 5.214   Mean   :3.413  
##  3rd Qu.:5.000   3rd Qu.:7.000   3rd Qu.: 6.000   3rd Qu.:5.000  
##  Max.   :8.000   Max.   :9.000   Max.   :10.000   Max.   :7.000  
##   bty_m1upper     bty_m2upper       bty_avg           pic_outfit 
##  Min.   :1.000   Min.   :1.000   Min.   :1.667   formal    : 77  
##  1st Qu.:3.000   1st Qu.:4.000   1st Qu.:3.167   not formal:386  
##  Median :4.000   Median :5.000   Median :4.333                   
##  Mean   :4.147   Mean   :4.752   Mean   :4.418                   
##  3rd Qu.:5.000   3rd Qu.:6.000   3rd Qu.:5.500                   
##  Max.   :9.000   Max.   :9.000   Max.   :8.167                   
##        pic_color  
##  black&white: 78  
##  color      :385  
##                   
##                   
##                   
## 
str(evals)
## 'data.frame':    463 obs. of  21 variables:
##  $ score        : num  4.7 4.1 3.9 4.8 4.6 4.3 2.8 4.1 3.4 4.5 ...
##  $ rank         : Factor w/ 3 levels "teaching","tenure track",..: 2 2 2 2 3 3 3 3 3 3 ...
##  $ ethnicity    : Factor w/ 2 levels "minority","not minority": 1 1 1 1 2 2 2 2 2 2 ...
##  $ gender       : Factor w/ 2 levels "female","male": 1 1 1 1 2 2 2 2 2 1 ...
##  $ language     : Factor w/ 2 levels "english","non-english": 1 1 1 1 1 1 1 1 1 1 ...
##  $ age          : int  36 36 36 36 59 59 59 51 51 40 ...
##  $ cls_perc_eval: num  55.8 68.8 60.8 62.6 85 ...
##  $ cls_did_eval : int  24 86 76 77 17 35 39 55 111 40 ...
##  $ cls_students : int  43 125 125 123 20 40 44 55 195 46 ...
##  $ cls_level    : Factor w/ 2 levels "lower","upper": 2 2 2 2 2 2 2 2 2 2 ...
##  $ cls_profs    : Factor w/ 2 levels "multiple","single": 2 2 2 2 1 1 1 2 2 2 ...
##  $ cls_credits  : Factor w/ 2 levels "multi credit",..: 1 1 1 1 1 1 1 1 1 1 ...
##  $ bty_f1lower  : int  5 5 5 5 4 4 4 5 5 2 ...
##  $ bty_f1upper  : int  7 7 7 7 4 4 4 2 2 5 ...
##  $ bty_f2upper  : int  6 6 6 6 2 2 2 5 5 4 ...
##  $ bty_m1lower  : int  2 2 2 2 2 2 2 2 2 3 ...
##  $ bty_m1upper  : int  4 4 4 4 3 3 3 3 3 3 ...
##  $ bty_m2upper  : int  6 6 6 6 3 3 3 3 3 2 ...
##  $ bty_avg      : num  5 5 5 5 3 ...
##  $ pic_outfit   : Factor w/ 2 levels "formal","not formal": 2 2 2 2 2 2 2 2 2 2 ...
##  $ pic_color    : Factor w/ 2 levels "black&white",..: 2 2 2 2 2 2 2 2 2 2 ...
CREATE A SCATTERPLOT FOR ‘AGE’ VS ‘BTY_AVG’
plot(evals$age, evals$bty_avg)

1
CREATE A BOXPLOT FOR ‘AGE’ AND ‘GENDER’
boxplot(evals$age ~ evals$gender)

2
CREATE A MOSAIC PLOT FOR ‘RANK’ AND ‘GENDER’
mosaicplot(evals$rank~evals$gender)

3
SIMPLE LINEAR REGRESSION
The fundamental phenomenon suggested by the study is that better looking teachers are evaluated more favorably.

Let’s create a scatterplot to see if this appears to be the case.

Instructions

Create a scatterplot for the average beauty rating of a professor bty_avg (x value) and score (y value).

Before drawing conclusions, compare the number of observations in the data frame with the approximate number
of points on the scatterplot. 

Do you notice anything awry?
CREATE A SCATTERPLOT FOR ‘SCORE’ AND ‘BTY_AVG’
plot( evals$bty_avg, evals$score)

4
The jitter function

Replot the scatterplot, but this time use the jitter() function. 

What was misleading about the initial scatterplot?
Instructions

Redo the scatterplot of the previous exercise, but use jitter() on the x or y coordinate. 

Review the documentation by executing ?jitter in the console.
APPLY ‘JITTER’ ON THE ‘BTY_AVG’ OR ‘SCORE’ VARIABLE OF YOUR INITIAL PLOT
plot(evals$score ~ jitter(evals$bty_avg))

5
plot(jitter(evals$score) ~ evals$bty_avg)

6
#jitter() adds a small amount of noise to a numeric vector which solved the overplotting problem.
More than natural variation?

Let's see if the apparent trend in the plot is something more than natural variation. 

Fit a linear model called m_bty to predict average professor score by average beauty rating
and add the line to your plot using abline(m_bty).
Instructions

Construct the linear model m_bty.
Add the linear model to the plot with the help of abline(m_bty).
Write out the equation for the linear model yourself, and interpret the slope.
# Initial plot
plot(evals$score ~ jitter(evals$bty_avg))

# Construct the linear model
m_bty <- lm(score ~ bty_avg, data = evals)

# Plot the linear model on the initial plot
abline(m_bty)

7
MULTIPLE LINEAR REGRESSION
The data set contains several variables on the beauty score of the professor: 

individual ratings from each of the six students who were asked to score the physical 
appearance of the professors and the average of these six scores.

Let's take a look at the relationship between one of these scores and the average 
beauty score.
Instructions

Create a scatterplot with the beauty rating of professor from lower level female bty_f1lower
and the average beauty score bty_avg.

Calculate and print the correlation between these two variables.
# Initial scatterplot
plot(evals$bty_f1lower ~ evals$bty_avg)

8
# The correlation
cor(evals$bty_f1lower, evals$bty_avg)
## [1] 0.8439112
# Indeed. As expected, the relationship is quite strong.
The relationship between all beauty variables.

As seen in the previous exercise, the relationship is quite strong; 
after all, the average score is calculated using the individual scores. 

We can actually take a look at the relationships between all beauty variables 
(columns 13 through 19) using the following command.
Instructions

Have a look at the relationships between all beauty variables (columns 13 through 19).
plot(evals[, 13:19])

9
These variables are collinear (correlated), and adding more than one of these variables
to the model would not add much value to the model. 

In this application and with these highly-correlated predictors, it is reasonable to use
the average beauty score as the single representative of these variables.
Taking into account gender

In order to see if beauty is still a significant predictor of professor score after we've
accounted for the gender is the professor, we can add the gender term into the model.
Instructions

Fit a linear model called m_bty_gen to predict average professor score by average beauty
rating bty_avg and by gender. 

You can use the + sign to add more explanatory variables to the model.
Study the outcome with the help of summary().
# The new linear model
m_bty_gen <- lm(score ~ bty_avg + gender, data = evals)

# Study the outcome
summary(m_bty_gen)
## 
## Call:
## lm(formula = score ~ bty_avg + gender, data = evals)
## 
## Residuals:
##     Min      1Q  Median      3Q     Max 
## -1.8305 -0.3625  0.1055  0.4213  0.9314 
## 
## Coefficients:
##             Estimate Std. Error t value Pr(>|t|)    
## (Intercept)  3.74734    0.08466  44.266  < 2e-16 ***
## bty_avg      0.07416    0.01625   4.563 6.48e-06 ***
## gendermale   0.17239    0.05022   3.433 0.000652 ***
## ---
## Signif. codes:  0 '***' 0.001 '**' 0.01 '*' 0.05 '.' 0.1 ' ' 1
## 
## Residual standard error: 0.5287 on 460 degrees of freedom
## Multiple R-squared:  0.05912,    Adjusted R-squared:  0.05503 
## F-statistic: 14.45 on 2 and 460 DF,  p-value: 8.177e-07
Look closely at the outcome of summary. 

Is bty_avg still a significant predictor of score?

Has the addition of gender to the model changed the parameter estimate for bty_avg?
P-Values

P-values and parameter estimates should only be trusted if the conditions for the 
regression are reasonable. 

Using diagnostic plots, can we conclude that the conditions for this model are reasonable?

Yes we can.
Gendermale

Note that the estimate for gender is called gendermale in your summary output. 

You'll see this name change whenever you introduce a categorical variable. 

The reason is that R recodes gender from having the values of female and male to 
being an indicator variable called gendermale that takes a value of 0 for females 
and a value of 1 for males. 
(Such variables are often referred to as 'dummy' variables.)

As a result, for females, the parameter estimate is multiplied by zero, leaving the
intercept and slope form familiar from simple regression.

score^ = b0 + b1 × bty_avg + b2 × (0)
       = b0 + b1 × bty_avg
Instructions

Plot this line and the line corresponding to males with the following custom function
multiLines(m_bty_gen)

10
# The plot
multiLines(m_bty_gen)
Try to determine the equation of the line corresponding to males yourself.

(Hint: For males, the parameter estimate is multiplied by 1.)

For your information, the decision to call the indicator variable gendermale instead
of genderfemale has no deeper meaning. 

R simply codes the category that comes first alphabetically as a 0.
SWITCHING RANK AND GENDER
Create a new model called m_bty_rank with gender removed and rank added in.
Instructions

Fit a linear model called m_bty_rank to predict average professor score by average 
beauty rating bty_avg and by rank (add the two variables).

Print the outcome with the help of summary().

How does R appear to handle categorical variables that have more than two levels? 

Note that the rank variable has three levels: teaching, tenure track, tenured.
# The linear model with rank and average beauty
m_bty_rank <- lm(score ~ bty_avg + rank, data = evals)

# View the regression output
summary(m_bty_rank)
## 
## Call:
## lm(formula = score ~ bty_avg + rank, data = evals)
## 
## Residuals:
##     Min      1Q  Median      3Q     Max 
## -1.8713 -0.3642  0.1489  0.4103  0.9525 
## 
## Coefficients:
##                  Estimate Std. Error t value Pr(>|t|)    
## (Intercept)       3.98155    0.09078  43.860  < 2e-16 ***
## bty_avg           0.06783    0.01655   4.098 4.92e-05 ***
## ranktenure track -0.16070    0.07395  -2.173   0.0303 *  
## ranktenured      -0.12623    0.06266  -2.014   0.0445 *  
## ---
## Signif. codes:  0 '***' 0.001 '**' 0.01 '*' 0.05 '.' 0.1 ' ' 1
## 
## Residual standard error: 0.5328 on 459 degrees of freedom
## Multiple R-squared:  0.04652,    Adjusted R-squared:  0.04029 
## F-statistic: 7.465 on 3 and 459 DF,  p-value: 6.88e-05
Since rank has three levels (teaching, tenure track, tenured) two indicator variables 
are created: 

one for tenure track and and one for tenured. 

Teaching is the reference level hence it doesn't show up in the regression output.
Brainstorming:

Which of the following is the correct order of the three levels of rank if we were to 
order them from lowest predicted course evaluation score to highest predicted course 
evaluation score?
Instructions

Teaching, Tenure Track, Tenured
Tenure track, Tenured
Tenure Track, Tenured, Teaching
Teaching, Tenured, Tenure Track
Well!

In general, the interpretation of the coefficients in multiple regression is slightly 
different from that of simple regression. 

The estimate for bty_avg reflects how much higher a group of professors is expected to 
score if they have a beauty rating that is one point higher while holding all other 
variables constant. 

In this case, that translates into considering only professors of the same rank with 
bty_avg scores that are one point apart.
THE SEARCH FOR THE BEST MODEL
You will start with a full model that predicts professor score based on rank, ethnicity, 
gender, language of the university where they got their degree, age, proportion of students
that filled out evaluations, class size, course level, number of professors, number of 
credits, average beauty rating, outfit, and picture color.

Note you do not included the pic_outfit or pic_color variables in the full model because 
the original study states that these variables were used in a different analysis evaluating
whether they're related to how highly the six students involved in the study score the 
professors' beauty (not related to how the students evaluate their professors in class).
Instructions

Before running the model, think about which variable you would expect to have the highest
p-value in this model and why.

Run the full model, the code is displayed in the editor.

Print the summary of the m_full model.
# The full model:
m_full <- lm(score ~ rank + ethnicity + gender + language + age + cls_perc_eval + cls_students + cls_level + cls_profs + cls_credits + bty_avg, data = evals)

# View the regression output:
summary(m_full)
## 
## Call:
## lm(formula = score ~ rank + ethnicity + gender + language + age + 
##     cls_perc_eval + cls_students + cls_level + cls_profs + cls_credits + 
##     bty_avg, data = evals)
## 
## Residuals:
##      Min       1Q   Median       3Q      Max 
## -1.84482 -0.31367  0.08559  0.35732  1.10105 
## 
## Coefficients:
##                         Estimate Std. Error t value Pr(>|t|)    
## (Intercept)            3.5305036  0.2408200  14.660  < 2e-16 ***
## ranktenure track      -0.1070121  0.0820250  -1.305 0.192687    
## ranktenured           -0.0450371  0.0652185  -0.691 0.490199    
## ethnicitynot minority  0.1869649  0.0775329   2.411 0.016290 *  
## gendermale             0.1786166  0.0515346   3.466 0.000579 ***
## languagenon-english   -0.1268254  0.1080358  -1.174 0.241048    
## age                   -0.0066498  0.0030830  -2.157 0.031542 *  
## cls_perc_eval          0.0056996  0.0015514   3.674 0.000268 ***
## cls_students           0.0004455  0.0003585   1.243 0.214596    
## cls_levelupper         0.0187105  0.0555833   0.337 0.736560    
## cls_profssingle       -0.0085751  0.0513527  -0.167 0.867458    
## cls_creditsone credit  0.5087427  0.1170130   4.348  1.7e-05 ***
## bty_avg                0.0612651  0.0166755   3.674 0.000268 ***
## ---
## Signif. codes:  0 '***' 0.001 '**' 0.01 '*' 0.05 '.' 0.1 ' ' 1
## 
## Residual standard error: 0.504 on 450 degrees of freedom
## Multiple R-squared:  0.1635, Adjusted R-squared:  0.1412 
## F-statistic: 7.331 on 12 and 450 DF,  p-value: 2.406e-12
Make sure to check your suspicions on which variable you expected to have the highest
p-value in this model.
Brainstorming

Which of the following is the correct interpretation of the coefficient associated 
with the ethnicity variable.

Non-minority professors are expected on average to score...

Note: your model m_full is still loaded in the workspace. Check the summary.
Instructions

0.19 points lower than minority professors, all else held constant.
0.19 points higher than minority professors, all else held constant. (correct)
0.02 points lower than minority professors, all else held constant.
0.02 points higher than minority professors, all else held constant.
ELIMINATING VARIABLES FROM THE MODEL – P-VALUE SELECTION
Now you will create a new model, where you will drop the variable with the highest
p-value in the m_full model.
Instructions

Create the model m_new, where the variable with the highest p-value of the m_full
is left out.

Have a look at the summary of the m_new model. 

Did the coefficients and significance of the other explanatory variables change?
# The full model
m_full <- lm(score ~ rank + ethnicity + gender + language + age + cls_perc_eval + cls_students + cls_level + cls_profs + cls_credits + bty_avg, data = evals)

# The new model
m_new <- lm(score ~ rank + ethnicity + gender + language + age + cls_perc_eval + cls_students + cls_level + cls_credits + bty_avg, data = evals)

# View the regression output
summary(m_new)
## 
## Call:
## lm(formula = score ~ rank + ethnicity + gender + language + age + 
##     cls_perc_eval + cls_students + cls_level + cls_credits + 
##     bty_avg, data = evals)
## 
## Residuals:
##      Min       1Q   Median       3Q      Max 
## -1.85048 -0.31394  0.08052  0.35956  1.10356 
## 
## Coefficients:
##                         Estimate Std. Error t value Pr(>|t|)    
## (Intercept)            3.5286297  0.2402990  14.684  < 2e-16 ***
## ranktenure track      -0.1073638  0.0819096  -1.311 0.190606    
## ranktenured           -0.0453744  0.0651169  -0.697 0.486278    
## ethnicitynot minority  0.1893718  0.0760992   2.488 0.013189 *  
## gendermale             0.1780270  0.0513581   3.466 0.000578 ***
## languagenon-english   -0.1265737  0.1079088  -1.173 0.241427    
## age                   -0.0066619  0.0030788  -2.164 0.031006 *  
## cls_perc_eval          0.0056790  0.0015448   3.676 0.000265 ***
## cls_students           0.0004493  0.0003573   1.257 0.209319    
## cls_levelupper         0.0183743  0.0554870   0.331 0.740687    
## cls_creditsone credit  0.5109162  0.1161614   4.398 1.36e-05 ***
## bty_avg                0.0611497  0.0166432   3.674 0.000267 ***
## ---
## Signif. codes:  0 '***' 0.001 '**' 0.01 '*' 0.05 '.' 0.1 ' ' 1
## 
## Residual standard error: 0.5035 on 451 degrees of freedom
## Multiple R-squared:  0.1635, Adjusted R-squared:  0.1431 
## F-statistic: 8.012 on 11 and 451 DF,  p-value: 8.303e-13
Note:

If you look at the summary, you see that the coefficients and p-values of some variables
changed. 

One of the things that makes multiple regression interesting is that coefficient estimates
depend on the other variables that are included in the model.
ELIMINATING VARIABLES FROM THE MODEL – ADJUSTED R-SQUARED SELECTION
Now you will create a new model, where you will drop the variable that when dropped yields
the highest improvement in the adjusted R2.
Instructions

Create a new model, m1, where you remove rank from the list of explanatory variables. 

Check out the adjusted R2 of this new model and compare it to the adjusted R2 of the full model.

If you don't want to view the entire model output, but just the adjusted R-squared, use 

summary(m1)$adj.r.squared.

Create another new model, m2, where you remove ethnicity from the list of explanatory variables.

Check out the adjusted R2 of this new model and compare it to the adjusted R2 of the full model.

Repeat until you have tried removing each variable from the full model m_full at a time, 
and determine the removal of which variable yields the highest improvement in the adjusted R-sqared.

Make note of this variable (you will be asked about it in the following session).
# The full model:
m_full <- lm(score ~ rank + ethnicity + gender + language + age + cls_perc_eval + cls_students + cls_level + cls_profs + cls_credits + bty_avg, data = evals)
summary(m_full)$adj.r.squared
## [1] 0.1412172
# Remove rank:
m1 <- lm(score ~ ethnicity + gender + language + age + cls_perc_eval + cls_students + cls_level + cls_profs + cls_credits + bty_avg, data = evals)
summary(m1)$adj.r.squared
## [1] 0.1417823
# Remove ethnicity:
m2 <- lm(score ~ rank + gender + language + age + cls_perc_eval + cls_students + cls_level + cls_profs + cls_credits + bty_avg, data = evals)
summary(m2)$adj.r.squared
## [1] 0.1320486
# Remove gender:
m3 <- lm(score ~ rank + ethnicity + language + age + cls_perc_eval + cls_students + cls_level + cls_profs + cls_credits + bty_avg, data = evals)
summary(m3)$adj.r.squared
## [1] 0.1202468
# Remove language:
m4 <- lm(score ~ rank + ethnicity + gender + age + cls_perc_eval + cls_students + cls_level + cls_profs + cls_credits + bty_avg, data = evals)
summary(m4)$adj.r.squared
## [1] 0.1404972
# Remove age:
m5 <- lm(score ~ rank + ethnicity + gender + language + cls_perc_eval + cls_students + cls_level + cls_profs + cls_credits + bty_avg, data = evals)
summary(m5)$adj.r.squared
## [1] 0.1342627
# Remove cls_perc_eval:
m6 <- lm(score ~ rank + ethnicity + gender + language + age + cls_students + cls_level + cls_profs + cls_credits + bty_avg, data = evals)
summary(m6)$adj.r.squared
## [1] 0.1174213
# Remove cls_students:
m7 <- lm(score ~ rank + ethnicity + gender + language + age + cls_perc_eval + cls_level + cls_profs + cls_credits + bty_avg, data = evals)
summary(m7)$adj.r.squared
## [1] 0.1401804
# Remove cls_level:
m8 <- lm(score ~ rank + ethnicity + gender + language + age + cls_perc_eval + cls_students + cls_profs + cls_credits + bty_avg, data = evals)
summary(m8)$adj.r.squared
## [1] 0.1429056
# Remove cls_profs:
m9 <- lm(score ~ rank + ethnicity + gender + language + age + cls_perc_eval + cls_students + cls_level + cls_credits + bty_avg, data = evals)
summary(m9)$adj.r.squared
## [1] 0.1430683
# Remove cls_credits:
m10 <- lm(score ~ rank + ethnicity + gender + language + age + cls_perc_eval + cls_students + cls_level + cls_profs + bty_avg, data = evals)
summary(m10)$adj.r.squared
## [1] 0.107127
# Remove bty_avg:
m11 <- lm(score ~ rank + ethnicity + gender + language + age + cls_perc_eval + cls_students + cls_level + cls_profs + cls_credits, data = evals)
summary(m11)$adj.r.squared
## [1] 0.1174189
Take some time to reflect on this final model. 

Based on your final model, what are the characteristics of a professor and course 
at University of Texas at Austin that would be associated with a high evaluation score? 

Would you be comfortable generalizing your conclusions to apply to professors 
generally (at any university)? Why or why not?


Elimination of which variable from the full model yielded the highest adjusted R-squared?

Instructions

  bty_avg
  cls_profs
  cls_students
  rank
