
%-------------------------------------------------%
\noindent \textbf{Variables in Regression Analysis }
\begin{itemize}
	\item The X variable is called the independent (or predictor) variable.
	\item The Y variable is called the dependent (or response) variable.
	\item Using the scatter plot we can state the strength and type
	(linear/non-linear) of the relationship.
\end{itemize}
%-------------------------------------------------%
\noindent \textbf{Correlation and cause-effect}
\begin{itemize}
	\item Note that a strong relationship between two variables does not
	imply a cause-effect relationship.
	\item For example, there is a strong negative correlation between the
	sales of ice cream and the number of flu infections.
	\item This does not mean that ice cream protects against flu.
	\item This relationship results from a latent variable (a variable that has
	not been observed).
	\item Such a latent variable in this case is the weather. Low
	temperatures and wet weather result in a high number of flu
	infections and low ice cream sales. \item Hot, sunny weather leads to the
	opposite.
\end{itemize}

%-------------------------------------------------%
\noindent \textbf{Scatter-plots}
Subsequent Slides
\begin{itemize}
	\item Relatively strong positive relationship (as height increases
	weight on average increases), reasonably linear.
	\item No relationship/weak negative relationship
	\item Negative, very strong, non-linear relationship.
	\item Non-linear relationship.
\end{itemize}
\section{What is Bivariate data?}


\begin{itemize}
	
	
	\item A dataset with two variables contains what is called bivariate data 
	\item For example, the heights and weights of people (i.e. for the purposes of determining the extent to which taller people weigh more)
	
\end{itemize}



\section{Univariate and  Bivariate Data}
Statistical data is often classified according to the number of variables being studied.
\begin{itemize}
	\item	\textbf{Univariate data}. When we conduct a study that looks at only one variable, we say that we are working with univariate data. Suppose, for example, that we conducted a survey to estimate the average weight of high school students. Since we are only working with one variable (weight), we would be working with univariate data.
	\item	\textbf{Bivariate data}. When we conduct a study that examines the relationship between two variables, we are working with bivariate data. Suppose we conducted a study to see if there were a relationship between the height and weight of high school students. Since we are working with two variables (height and weight), we would be working with bivariate data. 
\end{itemize}

