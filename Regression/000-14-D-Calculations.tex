\documentclass[]{report}

\voffset=-1.5cm
\oddsidemargin=0.0cm
\textwidth = 480pt

\usepackage{framed}
\usepackage{subfiles}
\usepackage{graphics}
\usepackage{newlfont}
\usepackage{eurosym}
\usepackage{amsmath,amsthm,amsfonts}
\usepackage{amsmath}
\usepackage{color}
\usepackage{amssymb}
\usepackage{multicol}
\usepackage[dvipsnames]{xcolor}
\usepackage{graphicx}
\begin{document}
%-------------------------------------------------%


\section{Regression Equation}
\begin{itemize}
\item A regression model is a statistical analysis assessing the association between two variables. It is used to find the relationship between two variables.
	
\item Linear Regression is a statistical technique that correlates the change in a variable (a series of data that recurs at fixed intervals) to other variable/s. 

\item The representation of the relationship is called the linear regression model. It is called linear because the relationship is linearly additive. 

\item Below is an example of a linear regression model:
	
	\[Y= a + bx + \epsilon\]
\end{itemize}




\section{Intercept Slope calculations}

We are asked to calculate the following
\begin{itemize} 
	\item an estimate for the intercept value
	\item an estimate for the slope value
\end{itemize}
(The chevron sign denotes that the value in question is an estimate.)


\section{Intercept Slope calculations}

We calculate the estimate for slope first.

To calculate the estimate for the intercept, we first must determine the values for the means of X and Y (i.e  and ). We are given the values of the summations of X and Y (i.e. and ), which we divide by the number of XY pairs (‘n’). 

We then construct the regression model equation , which estimate a value for Y for a  given X value. It takes the form:	

The second part of the question will give us a particular X value and ask us to calculate a corresponding estimate for Y.










\section{Part I Interpreting scatterplot}


This scatterplot suggests a weak positive linear relationship between the daily high temperatures and the year.

%Σ xi = 465;         Σ yi = 387.23
%SX,Y = 60.065;       SX,X = 2247.5;           SY,Y = 8.0033.

Part II 	Correlation


This Correlation value indicates weak positive linear relationship between temperatures and year.
%---------------------------------------------------------------------%
\section{Regression:Computing the Estimates}
\begin{itemize}
	\item Slope Estimate
	\[b_1 = \frac{S_{XY}}{S_{XX}} \]
	\item Intercept Estimate
	\[b_0 = \bar{Y} - b_1\bar{X} \]
\end{itemize}

\section{Regression: Computing the Estimates}
\begin{itemize}
	\item Slope Estimate
	\[b_1 = \frac{S_{XY}}{S_{XX}} \]
	\item Intercept Estimate
	\[b_0 = \bar{Y} - b_1\bar{X} \]
\end{itemize}

%-------------------------------------------------%

\end{document}
