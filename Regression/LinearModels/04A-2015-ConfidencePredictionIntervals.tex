Confidence Intervals and Prediction Intervals for Fitted Values
Previously we have seen the confidence intervals for regression coefficients in linear models. These confidence intervals are computed using the standard error values , which are available on the output of the summary() command, when using R.
Recall that confidence intervals are generally constructed using point estimates, quantiles and standard errors.
In this class we will look at two more type of intervals.
Confidence Intervals for Fitted Values Prediction Intervals for Fitted Values
Recall: a fitted value Y* is a estimate for the response variable, as determined by a linear model. The difference between the observed value and the corresponding fitted value is known as the residual.
The residual standard error is the conditional standard deviation of the dependent variable Y given a value of the independent variable X. The calculation of this standard error follows from the 
definition of the residuals.


%===========================================================================================%

%- http://www.psychstat.missouristate.edu/introbook/sbk16.htm

\subsection*{Computational Formula}


The calculation of the standard error of estimate is simplified by the following formula, called the computational formula for the standard error of estimate. 
The computation is easier because the statistical calculator computed the correlation coefficient when finding a regression line. 
The computational formula for the standard error of estimate will always give the same result, within rounding error, as the definitional formula. 
The computational formula may look more complicated, but it does not require the computation of the entire table of differences between observed and predicted Y scores. The computational formula is as follows:




%===========================================================================================%
%-  http://people.stat.sc.edu/hendrixl/stat205/Lecture%20Notes/Regression%20and%20Correlation.pdf

\subsection{Conditional Mean and Standard Deviation}


Definition:  A conditional mean is the expected value of a variable conditional on another
variable.
Notation: μY|X
Defiinition:  A conditional standard deviation is the standard deviation of a variable
conditional on another variable.
Notation: σY|X


%===========================================================================================%


Recall : the regression equation is
Fluo* = 1.518 + 1.930 Conc
(The asterisk denotes a fitted value, as opposed to an observed value. We use this due to typographical limitations in producing PDFs)
The R command we will use is predict(). We are going to predict the fluorescence values for new values of concentration (written in the code as Conc+1, essentially adding 1 to each value of the independent variable). We specify the following: The name of the fitted model (essentially stating what regression equation to use) The new data set ( which must be constructed as an object known as “data.frame” – we will cover data frames in class soon).
o If we are using the original data set, we can leave the argument blank.
o If we are using new data, we must specify the name of the new data as that of the original independent variable. We must specify what type of interval we require (
i.e. confidence interval or prediction interval)

Let’s try this out with the original data first. 
> predict(FitFC, interval="confidence") 

Now let’s try this with new data. 
> Conc [1] 0 2 4 6 8 10 12 


Lets create some new data. To keep it simple we just use even value numbers.
\begin{framed}
\begin{verbatim}

> Conc+1 #The New Data [1] 1 3 5 7 9 11 13 
> 
> #Save it as a dataframe 
> #Use the same column names as the original Ind. Variables. 
> new <- data.frame(Conc=c(Conc+1)) 
\end{verbatim}
\end{framed}


> 
> predict(lm(Fluo ~ Conc), new, interval="confidence") 