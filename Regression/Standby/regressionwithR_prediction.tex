
\documentclass[a4paper,12pt]{article}
%%%%%%%%%%%%%%%%%%%%%%%%%%%%%%%%%%%%%%%%%%%%%%%%%%%%%%%%%%%%%%%%%%%%%%%%%%%%%%%%%%%%%%%%%%%%%%%%%%%%%%%%%%%%%%%%%%%%%%%%%%%%%%%%%%%%%%%%%%%%%%%%%%%%%%%%%%%%%%%%%%%%%%%%%%%%%%%%%%%%%%%%%%%%%%%%%%%%%%%%%%%%%%%%%%%%%%%%%%%%%%%%%%%%%%%%%%%%%%%%%%%%%%%%%%%%
\usepackage{eurosym}
\usepackage{vmargin}
\usepackage{amsmath}
\usepackage{graphics}
\usepackage{framed}
\usepackage{epsfig}
\usepackage{subfigure}
\usepackage{fancyhdr}

\setcounter{MaxMatrixCols}{10}
%TCIDATA{OutputFilter=LATEX.DLL}
%TCIDATA{Version=5.00.0.2570}
%TCIDATA{<META NAME="SaveForMode"CONTENT="1">}
%TCIDATA{LastRevised=Wednesday, February 23, 201113:24:34}
%TCIDATA{<META NAME="GraphicsSave" CONTENT="32">}
%TCIDATA{Language=American English}

\pagestyle{fancy}
\setmarginsrb{20mm}{0mm}{20mm}{25mm}{12mm}{11mm}{0mm}{11mm}
\lhead{MA4128} \rhead{Kevin O'Brien} \chead{Assumptions for Linear Models} %\input{tcilatex}

\begin{document}

\section{Prediction}
Suppose we want to know the predicted value $\hat{y}$ at x = 30. Could
write out the equation using the parameter estimates and fill in the
required value for x:
\begin{framed}
\begin{verbatim}
>19.94379 + 2.07497*30
[1] 82.19289
\end{verbatim}
\end{framed}
Alternatively, could use the predict function:
\begin{framed}
\begin{verbatim}
predict(model,list(x=30))
1
82.19289
\end{verbatim}
\end{framed}
For linear plots use abline for superimposing the model on a
scatterplot of the data points. For curved responses, use the
predict function to generate the lines.

\subsection{Confidence and Prediction Intervals}
Fitted lines are often presented with uncertainty bands around
them. There are two types of bands:
\begin{itemize}
\item Confidence bands - refer to the POPULATION.
\item Prediction bands - refer to an INDIVIDUAL.
\end{itemize}

One way to get intervals:
\begin{framed}
\begin{verbatim}
predict(model, interval="confidence")
predict(model, interval="prediction")
\end{verbatim}
\end{framed}

In order plot the bands on the same plot as the fitted line, use the
following:
\begin{framed}
\begin{verbatim}
grid <- seq(5,50)
pi <- predict(model, list(x=grid), interval="prediction")
ci <- predict(model, list(x=grid), interval="confidence")
plot(SLR1$x, SLR1$y, xlab="x", ylab="y")
matlines(grid, pi, lty=c(1,2,2), col=c("black", "blue","blue"))
matlines(grid, ci, lty=c(1,3,3), col=c("black", "red","red"))
legend("bottomright", legend=c("Prediction Interval", "Confidence Interval"),
lty=2:3, col=c("blue","red"), bty="n")
\end{verbatim}
\end{framed}


\end{document}
